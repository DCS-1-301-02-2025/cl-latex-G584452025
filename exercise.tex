\documentclass[a4paper,11pt,dvipdfmx]{ujarticle}
% パッケージ
\usepackage{graphicx}
\usepackage{url}
% レイアウト指定を記述したファイルの読み込み
\input{layout}

% タイトルと氏名を変更せよ.
\title{日本におけるデジタル化の状況}
\author{G584452025 櫻井 暖真}

\begin{document}

\maketitle %ここにタイトルが入る

% ここから本文
\section{デジタル競争力ランキング}


国際経営開発研究所(IMD)の調査 \cite{imd} によると、
日本のデジタル競争力のランキングは図\ref{fig:ranking} に示すように、
調査対象の64カ国中、総合で28位、知識項目で25位となっている。

\begin{figure}[htbp]
    \centering
    \includegraphics[width=1\textwidth]{fig31.png} % 実際の画像に差し替えてください
    \caption{デジタル競争力ランキング(64カ国中)}
    \label{fig:ranking}
\end{figure}

\section{ブローバンドの整備状況}
OECDによるブロードバンド回線の普及に関する調査 \cite{oecd} によると、
表\ref{tab:broadband} に示すように、
日本における100人あたりのモバイルブロードバンドの加入者数は190.5で、
第1位となっている。2位はエストニアで、3位は韓国と続く。

\begin{table}[htbp]
    \centering
    \caption{モバイルブロードバンドの加入者数(100人あたり)}
    \label{tab:broadband}
    \begin{tabular}{|c|l|c|}
        \hline
        順位 & 国名 & 加入者数 \\
        \hline
        1位 & 日本 & 190.5 \\
        \hline
        2位 & エストニア & 170.9 \\
        \hline
        3位 & 米国 & 169.0 \\
        \hline
        4位 & フィンランド & 157.0 \\
        \hline
        5位 & デンマーク & 141.7 \\
        \hline
        6位 & ラトビア & 141.6 \\
        \hline
        7位 & イスラエル & 139.9 \\
        \hline
        8位 & オランダ & 133.7 \\
        \hline
        9位 & ポーランド & 131.3 \\
        \hline
        10位 & スウェーデン & 127.2 \\
        \hline
    \end{tabular}
\end{table}

\section{考察}
\begin{itemize}
    \item 日本はモバイルブロードバンドの普及率が非常に高い
    \item インフラ面では世界トップレベルにあることがわかる。
    \item デジタル競争力は他国と比べて低く、特に知識や技術の活用面に課題があると考えられる。
\end{itemize}

% を使う

% 本文(1)
%  参考文献の参照: \cite{}
%  図番号の参照: \ref{}
% を使う
% 文献データベースのキーワードは oecd と imd
% になっている.

% 図の挿入
% \includegraphics{}
% を
% \begin{figure}[htbp]
% \end{figure}
% で囲み
% \caption{}
% で図のタイトルを入れる.
% \label{}
% を使って図番号が参照できるようにする
% また,
% \centering
% で図が中央に来るようにする

% ーーー
% 節見出し(2)

% 本文(2)

% 表の挿入
% \begin{tabular}
% \end{tabular}    
% による表の記述を 
% \begin{table}[htbp]
% \end{table}
% で囲み
% \caption{}
% で表のタイトルを入れる.
% \label{}
% を使って表番号が参照できるようにする
% また,
% \centering
% で表が中央に来るようにする

% ーーー
% 見出し(3)
% 考察
%
% \begin{itemize}
% \end{itemize}
% を使って箇条書きで記述する

% ここに参考文献が入る
%
\bibliographystyle{junsrt}
\bibliography{exercise.bib}

\end{document}